Figure \ref{img:block_sdr_dongle} shows a block
diagram of one of the receiver dongles that where used
in this project.
The two main components on the dongle are a
\texttt{R820T2} tuner IC
and a \texttt{RTL2832U} DVB-T receiver IC. \\

\figurizefile{diagrams/sdr_clocks.tex}
             {img:block_sdr_dongle}
             {Block diagram of an unmodified SDR-Dongle}
             {0.7}{ht}

The tuner IC contains circuitry to drive an
external crystal for clock generation.
The crystal used on the hardware at hand
operates at a frequency of \SI{28.8}{\mega\hertz}.
The tuner uses this frequency to derive the
local oscillator signal used for mixing down
the RF-signal using a \gls{pll}.
It also provides a clock output pin that
outputs a triangle waveform at the crystals
oscillation frequency \cite{r820t2datasheet}. \\

The given USB-dongles use this clock output signal
as a clock input for the \texttt{RTL2832U}
receiver IC that handles the analog to digital
conversion and USB-communication.
The \gls{adc} sampling frequency is also derived from
the tuner clock output by using a phase locked loop. \\

In the basic setup four SDR dongles are used
to operate the antenna array.
In order to be able to compare the signal phases
received by these dongles they have to be tightly
synchronized in sample-aquisition time and frequency. \\

Gaining sufficient synchronization for DOA applications
by compensating for hardware impairments completely
in software was deemed unfeasible, so a hybrid
approach was choosen that requires minimal hardware
modifications and offloads most synchronization tasks
to the software side.
As demonstrated in 2013 by Juha Vierinen for
coherent radar applications \cite{rtlradar}. \\

Figure \ref{img:block_sdr_dongle_chained} shows
a schematic representation of the performed modifications.
The clock output of one tuner IC is daisy chained to
the clock input of the next tuner IC.
This way all tuners derive their clock frequency
from the same crystal, while the load on any of
the clock outputs is kept minimal.

To provide the input pin with the correct signal
levels a DC blocking RC network has to be used
on the input pins as shown in
figure \ref{img:block_sdr_dongle_chained}.

\figurizefile{diagrams/sdr_clocks_chained.tex}
             {img:block_sdr_dongle_chained}
             {Block diagram of a chained SDR-Dongle}
             {0.7}{ht}

The modifications that were actually performed on the
hardware can be seen in figure \ref{img:hw_mods}.
The dongle that contains the clock source for
the rest of the system can be seen in the top left
image, it still contains the original
\SI{28.8}{\mega\hertz} crystal and was only equipped
with thin wires that tap off the clock out signal
and a ground reference. \\

The next image to the right shows a dongle
in a later position in the clock distribution chain.
The original crystal was removed an replaced with
the RC-network shown in figure \ref{img:block_sdr_dongle_chained}.
The network is fed by the previos dongle in the chain. \\

The bottom right image in the figure shows the
locations of the main components on the dongles.
The signal path on the dongles is from the top right
to the left.
The signal enters the receiver through the antenna
connector at the top right, is mixed-down by the
tuner-IC and digitized by the \acrshort{adc} on the left.

\figurizefile{diagrams/annotated_hw_mods.tex}
             {img:hw_mods}
             {Hardware modifications to synchronize the clocks}
             {0.99}{ht}
