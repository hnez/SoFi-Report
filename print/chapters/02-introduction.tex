Direction of arrival estimation of \acrshort{rf}-Signals
works by analyzing the signals received by
multiple antennas distributed in space.

The common approach is to exploit the fact
that the signals received at antennas further
away from a source are more affected by the channel
they are transmitted over.

The effect of the channel can be measured by
observing the signal strength and/or phase
at the receiving antennas.

For systems where the distance between
the sender and the receiving antennas is much
smaller than the distance between the receiving
antennas, the difference in signal strength
are usually very small and using the phase
differences provides better accuracy. \\

The system implemented in this thesis will use \gls{sdr}
technology and fourier transformation to calculate \gls{doa}
information for all signals present in a given frequency range.
More sophisticated algorithms like \acrshort{music} employ
a cross-correlation Matrix as an early processing step,
allowing for better separation of signals and noise
but also limiting the number of signals analyzed at
a time to one fewer than the number of receivers. \\

Using multiple cheap, independent receivers instead
of purpose-built hardware also poses some challenges
that have to be overcome and will be discussed in
this thesis, like different LO frequencies and differences
in sampling time.

\begin{subchapter}{\gls{doa} estimation}
  Figure \ref{img:phased_array} visualizes how signal phases
  can be used to estimate the \acrlong{doa} of a signal
  in the far field. \\

  The diagram shows four antennas spaced out on a plane
  and a stilized wave that is received by all four of them.
  The source of the wave is assumed to be so far away that
  the signal-rays are approximately parallel when
  they reach the receivers. \\

  In the depicted constellation the signal has to travel the
  same distance to reach the receivers $A$ and $B$, thus the
  phases at those receivers are the same.
  On the other hand the signal has to travel further to reach
  receiver $D$ that to reach $A$, the difference in phase can
  be directly calculated from the wavelength of the signal and
  the distance between the antennas.

  \begin{equation*}
    \Delta \varphi =  2 \pi \cdot \frac{d_\text{AD}}{\lambda}
  \end{equation*}

  \figurizefile{diagrams/phased_array.tex}
               {img:phased_array}
               {Array of four antennas with a sender to the far right}
               {0.7}{ht}

  In a more realistic scenario the the signals will reach the
  receivers at odd angles. Figure \ref{img:angle_two_receivers}
  shows an example where a signal reaches the receivers at an
  angle of $\alpha=\SI{45}{\degree}$. The length that can be calculated
  from the phase difference $\Delta \varphi$ between the receivers and
  the wavelength $\lambda$ of the signal is marked as $\Delta l$.

  The direction of arrival $\alpha$ can then be calculated from
  those lengths using equation \ref{eq:doa_trigonometry}.

  \begin{equation}
    \label{eq:doa_trigonometry}
    \alpha
    = \arccos \left( \frac{\Delta l}{d} \right)
    = \arccos \left( \frac{\Delta \varphi \cdot \lambda}{2 \pi d} \right)
  \end{equation}

  \figurizefile{diagrams/angle_two_receivers.tex}
               {img:angle_two_receivers}
               {Signal arriving at an angle of \SI{45}{\degree}}
               {0.7}{ht}
\end{subchapter}

\begin{subchapter}{The effect of noise}
  There is however a mayor drawback to the DOA-estimation
  technique discussed above: it does not work in the presence of noise. \\

  This is because noise can be assumed to have no distinct source.
  If one were to inspect the signal phases in a system modeled after
  the block diagram in figure \ref{img:preprocessing_chain} before
  the averaging step, one would find that they look indeed quite random
  as to be expected from noise.
  But upon averaging the effects of the noise cancel out as can be seen in figure
  \ref{img:annotated_fft_phase_zoom} where the phases for frequencies
  without a strong signal are very close to zero. \\

  This means, that in the presence of noise the measured phase differences
  will always be smaller than those assumed in figure \ref{img:angle_two_receivers}
  by a factor $f$.
  Luckyly the effect of the noise can be assumed to be equal for
  all antenna pairs and thus the factor $f$ can also be assumed to be equal. \\

  Thus the algorithm used for direction estimation
  should be able to work relative with phases that are
  scaled by a common noise factor.
\end{subchapter}

% TODO: rip doa algorithm from software implementation
%       chapter and put it here

\begin{subchapter}{Receiver offsets: frequency}
  \Gls{sdr} devices are commonly structured as seen in
  figure \ref{img:block_sdr_dongle}, the antenna signal
  $S_\text{in}$ is first amplified by a factor of
  $A_\text{lna}$ by an \acrshort{lna} and then mixed
  down using an \acrshort{iqsig} - \acrshort{losig} signal
  $S_\text{lo}$ to form a complex baseband signal
  $S_\text{bb}$. \\

  \begin{equation}
    \label{eq:lomixer}
    S_\text{bb}
    = S_\text{in} \cdot A_\text{lna} \cdot
      \left( S_\text{lo,i} + i S_\text{lo,q} \right)
    = S_\text{in} \cdot A_\text{lna} \cdot
      e^{i \cdot \left( 2 \pi f_\text{lo} t + \varphi_\text{lo} \right)}
  \end{equation}

  To an user of an \gls{sdr} device the actual waveform
  of the \gls{losig} is usually irrelevant and it can be
  modelled as a complex sinusiod of frequency $f_\text{lo}$
  and phase $\varphi_\text{lo}$, as seen in equation \ref{eq:lomixer}. \\

  As discussed earlier \gls{doa} estimation uses the phase
  differences between signals received at different receivers
  to analyze the difference in signal runtime.

  A phase difference between two complex signals can be
  calculated using complex conjugate multiplication as
  shown in \ref{eq:ccmphdiff}.

  \begin{align}
    \label{eq:ccmphdiff}
    A \cdot e^{i \cdot \Delta \varphi}
    &= S_\text{bb,1} \cdot S_\text{bb,2}^\ast \\
    &= S_\text{in,1} \cdot
       e^{i \cdot \left( 2 \pi f_\text{lo,1} t + \varphi_\text{lo,1} \right)}
       \cdot
       S_\text{in,2}^\ast \cdot
       e^{-i \cdot \left( 2 \pi f_\text{lo,2} t + \varphi_\text{lo,2} \right)} \nonumber \\
    &= S_\text{in,1} \cdot S_\text{in,2}^\ast \cdot
       e^{i \cdot \left(
         2 \pi (f_\text{lo,1} - f_\text{lo,2}) t
         + \varphi_\text{lo,1} - \varphi_\text{lo,2}
       \right)} \nonumber \\
    \label{eq:ccmpoff}
    &= S_\text{in,1} \cdot S_\text{in,2}^\ast \cdot
       e^{i \cdot \left(
         2 \pi \Delta f_\text{lo} t
         + \Delta \varphi_\text{lo}
       \right)}
    % Convoluted equations in latex?
    % Still a better read than twilight
  \end{align}

  If the two signals are baseband signals that
  were downconverted using two \gls{losig} signals
  with respective frequencies of $f_\text{lo,1}$ and
  $f_\text{lo,2}$ and phases of $\varphi_\text{lo,1}$
  and $\varphi_\text{lo,2}$ the difference between
  these frequencies and phases will also be present
  in the resulting phase difference $\Delta \varphi$. \\

  In order to get the actual phase difference between
  the two input signals $S_\text{in,1}$ and $S_\text{in,2}$
  the \gls{losig} frequencies and phases have to be
  locked ($\Delta f_\text{lo} = 0$, $\Delta \varphi_\text{lo} = 0$).
\end{subchapter}
