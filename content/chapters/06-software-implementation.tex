The current implementation consists of three
programs that are connected using UNIX pipes
to form a DOA processing chain.

The components are: \\

- \texttt{backend/sofi} - Performs the sample-exact
timing offset compensation and the processing steps
shown in figure \ref{img:preprocessing_chain}. \\

- \texttt{frontend/sofi.py} - Performs further preprocessing
and and calculates direction of arrival estimations. \\

- \texttt{frontend/visualize.grc} - Gnuradio flowchart
that displays the vectors calculated by the frontend code. \\

\begin{subchapter}{Preprocessing}
  As the preprocessing code has to deal with samples
  before the first downsamling stage the data rates to be
  processed are rather high.
  In order to be able to run on low-range computing hardware
  this part is written in the C programming language and
  uses the highly optimized \texttt{fftw3} and \texttt{volk}
  libraries, for accelerated FFT and vector operations, respectively. \\

  \begin{lstlisting}[language=C]
    
  \end{lstlisting}
\end{subchapter}

\begin{subchapter}{DOA-estimation}
\end{subchapter}

\begin{subchapter}{Visualization}
\end{subchapter}
