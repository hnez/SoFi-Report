How phase differences between different receivers
can be used to calculate the direction of arrival of an
input wave can best be described in a diagram like
figure \ref{img:phased_array}.

The diagram shows four antennas spaced out on a plane
and a stilized wave that is received by all four of them.
The source of the wave is assumed to be so far away that
the rays are approximately parallel when they reach the receivers. \\

In the depicted constellation the signal has to travel the
same distance to reach the receivers $A$ and $B$, thus the
phase at those receivers is the same.
On the other hand the signal hat to travel further to reach
receiver $D$ that to reach $A$, the difference in phase can
be directly calculated from the wavelength of the signal and
the distance between the antennas.

\figurizefile{diagrams/phased_array.tex}
             {img:phased_array}
             {Array of four antennas with a sender to the far right}
             {0.7}

In a more realistic scenario the the signals will reach the
receivers at odd angles. Figure \ref{img:angle_two_receivers}
shows an example where a signal reaches the receivers at an
angle of $\alpha=\SI{45}{\degree}$. The length that can be calculated
from the phase difference $\Delta \varphi$ between the receivers and
the wavelength $\lambda$ of the signal is marked as $\Delta l$.

The direction of arrival $\alpha$ can then be calculated from
those lengths using equation \ref{eq:doa_trigonometry}.

\begin{equation}
  \label{eq:doa_trigonometry}
  \alpha
  = \arccos \left( \frac{\Delta l}{d} \right)
  = \arccos \left( \frac{\Delta \varphi \cdot \lambda}{2 \pi d} \right)
\end{equation}

\figurizefile{diagrams/angle_two_receivers.tex}
             {img:angle_two_receivers}
             {Signal arriving at an angle of \SI{45}{\degree}}
             {0.7}
