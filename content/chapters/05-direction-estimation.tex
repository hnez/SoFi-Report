\begin{subchapter}{From phase to angle}
  How phase differences between different receivers
  can be used to calculate the direction of arrival of an
  input wave can best be described in a diagram like
  figure \ref{img:phased_array}.

  The diagram shows four antennas spaced out on a plane
  and a stilized wave that is received by all four of them.
  The source of the wave is assumed to be so far away that
  the signal-rays are approximately parallel when they reach the receivers. \\

  In the depicted constellation the signal has to travel the
  same distance to reach the receivers $A$ and $B$, thus the
  phase at those receivers is the same.
  On the other hand the signal has to travel further to reach
  receiver $D$ that to reach $A$, the difference in phase can
  be directly calculated from the wavelength of the signal and
  the distance between the antennas.

  \figurizefile{diagrams/phased_array.tex}
               {img:phased_array}
               {Array of four antennas with a sender to the far right}
               {0.7}{ht}

  In a more realistic scenario the the signals will reach the
  receivers at odd angles. Figure \ref{img:angle_two_receivers}
  shows an example where a signal reaches the receivers at an
  angle of $\alpha=\SI{45}{\degree}$. The length that can be calculated
  from the phase difference $\Delta \varphi$ between the receivers and
  the wavelength $\lambda$ of the signal is marked as $\Delta l$.

  The direction of arrival $\alpha$ can then be calculated from
  those lengths using equation \ref{eq:doa_trigonometry}.

  \begin{equation}
    \label{eq:doa_trigonometry}
    \alpha
    = \arccos \left( \frac{\Delta l}{d} \right)
    = \arccos \left( \frac{\Delta \varphi \cdot \lambda}{2 \pi d} \right)
  \end{equation}

  \figurizefile{diagrams/angle_two_receivers.tex}
               {img:angle_two_receivers}
               {Signal arriving at an angle of \SI{45}{\degree}}
               {0.7}{ht}
\end{subchapter}

\begin{subchapter}{The effect of noise}
  There is however a mayor drawback to the DOA-estimation
  technique discussed above: it does not work in the prence of noise. \\

  This is because noise can be assumed to have no distinct source.
  If one were to inspect the signal phases in a system modeled after
  the block diagram in figure \ref{img:preprocessing_chain} before
  the averaging step, one would find that they look indeed quite random
  as to be expected from noise.
  But upon averaging the effects of the noise cancel out as can be seen in figure
  \ref{img:annotated_fft_phase_zoom} where the phases for frequencies
  without a strong signal are very close to zero. \\

  This means, that in the presence of noise the measured phase differences
  will always be smaller than those assumed in figure \ref{img:angle_two_receivers}
  by a factor $f$.
  Luckyly the effect of the noise can be assumed to be equal for
  all antenna pairs and thus the factor $f$ can also be assumed to be equal. \\

  Thus the algorithm used for direction estimation
  should be able to work relative phases that are
  scaled by a constant noise factor.
\end{subchapter}
